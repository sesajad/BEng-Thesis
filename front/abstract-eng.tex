
% -------------------------------------------------------
%  English Abstract
% -------------------------------------------------------


\pagestyle{empty}

\begin{latin}

\begin{center}
\textbf{Abstract}
\end{center}
\baselineskip=.8\baselineskip

Quantum computing is a computational model based on quantum mechanics principles. After introducing the Grover algorithm and Shor algorithm, quantum computing had become a trend in both theoretical and experimental fields. On the other hand, computational geometry is a branch of computer science that analyses geometrical problems from computational prospectives, like algorithms, complexity classes, and orders. By emerging these two fields, those problems could also be analyzed in the quantum model. Efforts in this emerging field had begun with the trend and are continued till today, but almost all of the efforts were done in Grover-based speedups that are maximum quadratic. The point-In-Polygon problem which is a useful problem in computational geometry and computer graphics is not studied yet in the quantum model but it's well-studied in the classical regime with a few linear algorithms that and tight bounds on the complexity. This thesis introduces a new algorithm, based on quantum Fourier transform, that in with a promise of distance from edges, achieves a superpolynomial speedup and solves the problem just with a query, but in the general case, it comes with no speed up and it's also proved that no algorithm can do such.

\bigskip\noindent\textbf{Keywords}:
Quantum Computing, Computational Geometry, Winding Number, Point in Polygon, Quantum Fourier Transform

\end{latin}

\newpage
