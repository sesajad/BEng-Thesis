
% -------------------------------------------------------
%  Abstract
% -------------------------------------------------------


\pagestyle{empty}

\شروع{وسط‌چین}
\مهم{چکیده}
\پایان{وسط‌چین}
\بدون‌تورفتگی
رایانشِ کوانتومی گونه‌ای از رایانش است که مبتنی بر اصل موضوع‌های مکانیک کوانتومی به وجود آمده 
که در دههٔ نود و پس از معرفیِ دو الگوریتمِ جست‌وجو و تجزیهٔ عدد، توجه‌ها به آن افزایش یافت.
از سوی دیگر، هندسهٔ محاسباتی که به بررسیِ مسائلِ هندسی از منظرِ رایانشی می‌پردازد و پیچیدگی و 
الگوریتم‌های این مسائل را بررسی می‌کند، می‌تواند در سیاقِ رایانشِ کوانتومی نیز موردِ بررسی قرار 
بگیرد. از بدو پیدایشِ رایانشِ کوانتومی بررسی‌های اندکی برروی کاربردِ آن در هندسهٔ محاسباتی صورت
گرفته که آن بررسی‌ها هم اکثراً به شکلِ استفاده از الگوریتمِ جست‌وجوی کوانتومی برروی مسائل بوده‌است
که با تسریعِ چندجمله‌ای همراه است. مسئلهٔ قرارگیریِ نقطه در چندضلعی که یکی از مسائلِ پرکاربردِ این حوزه
است که تا کنون به شکلِ کوانتومی بررسی نشده و با رایانشِ کلاسیک چند الگوریتم با زمانِ خطی برای آن
وجود دارد در کنار الگوریتم‌های تقریبی و با پیش‌پردازشی که در زمان و پرسش زیرخطی اجرا می‌شوند.
در این پایان‌نامه به معرفیِ الگوریتمی کوانتومی برای نقطه در چندضلعی می‌پردازیم که مبتنی بر تبدیلِ
فوریهٔ کوانتومی‌ست و می‌تواند در  شرایطی که تضمینِ فاصلهٔ نقطه از اضلاع وجود دارد به تسریعِ
فرا-چندجمله‌ای دست می‌یابد و با یک پرسش در زمانِ لگاریتمی می‌تواند پاسخ مسئله را به دست بیاورد
اما در حالتِ کلی تسریعی نسبت به حالتِ کلاسیک نخواهد داشت.

\پرش‌بلند
\بدون‌تورفتگی \مهم{کلیدواژه‌ها}: 
رایانشِ کوانتومی، هندسهٔ محساباتی، نقطه-در-چندضلعی، تبدیلِ فوریهٔ کوانتومی
\صفحه‌جدید
