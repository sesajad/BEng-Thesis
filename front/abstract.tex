
% -------------------------------------------------------
%  Abstract
% -------------------------------------------------------


\pagestyle{empty}

\شروع{وسط‌چین}
\مهم{چکیده}
\پایان{وسط‌چین}
\بدون‌تورفتگی
از بدوِ پیدایشِ رایانشِ کوانتومی، بررسی‌های اندکی برروی کاربردِ 
آن در هندسهٔ محاسباتی صورت گرفته که آن بررسی‌ها هم اکثراً 
معطوف به استفاده از الگوریتمِ جست‌وجوی کوانتومی برروی مسائل بوده‌است.
در این‌جا پس از مرورِ ادبیات، به معرفیِ الگوریتمی کوانتومی برای 
نقطه در چندضلعی می‌پردازیم که مبتنی بر تبدیلِ فوریهٔ کوانتومی‌ست 
و می‌تواند در  شرایطی که تضمینِ فاصلهٔ نقطه از اضلاع وجود دارد تنها با یک 
پرسش می‌تواند به جوابِ مسئله دست پیدا کند، در غیر این صورت تسریعی
ندارد و با همان $n$ پرسش، مشابه حالت کلاسیک عمل می‌کند.
\پرش‌بلند
\بدون‌تورفتگی \مهم{کلیدواژه‌ها}: 
پایان‌نامه، حروف‌چینی، قالب، زی‌پرشین
\صفحه‌جدید
