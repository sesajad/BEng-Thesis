
\فصل{نتیجه‌گیری}

در این پایان‌نامه سعی شد که به بررسیِ رایانشِ کوانتومی و هندسهٔ محاسباتی و تلاقیِ این دو حیطه پرداخته‌شود. آن‌چه به نظر می‌رسد این است که تلاش‌های کمی در ترکیبِ این دو حوزه صورت گرفته حال آن‌که به علتِ شکلِ مسائل در هندسه و وجودِ قول‌های متعددی که به خاطرِ ذاتِ هندسیِ مسائل، پتانسیل‌های زیادی برای بررسیِ مسائل و البته جمع‌بندی و تعمیمِ راه‌حل‌ها و الگوریتم‌ها وجود دارد.

از سوی دیگر، همان‌طور که گفته‌شد، اکثرِ تلاش‌هایی که برای طراحیِ الگوریتم و حدود صورت گرفته‌است مبتنی بر استفاده از الگوریتمِ جست‌وجو بوده‌اند و از این رو، اکثرِ نتایج بهبودهای مربعی و چندجمله‌ای هستند. اما این کار از جهتِ استفاده از الگوریتم‌هایی با تسریعِ فرا-چندجمله‌ای (که مبتنی بر تبدیلِ فوریهٔ کوانتومی هستند) می‌تواند متفاوت و مفید قلمداد شود.

\قسمت{کارهای آتی}
