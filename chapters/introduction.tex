
\فصل{مقدمه}

از مطرح شدنِ ایدهٔ رایانهٔ کوانتومی توسطِ فاینمن \مرجع{feynman} بیش از نیم قرن می‌گذرد اما عمدهٔ توجه‌ها به آن پس از دو الگوریتمِ مشهورِ شور \مرجع{shor} و گروور \مرجع{grover} در دههٔ نودِ میلادی بوده‌است. از آن روز تلاشِ بی‌وقفه‌ای برای یافتنِ الگوریتم‌های جدید و کاربردهای جدید از الگوریتم‌های قدیمیِ رایانشِ کوانتومی ادامه دارد. در همان سال‌ها نخستین آزمایش‌های موفقِ مربوط به رایانشِ کوانتومی انجام گرفت \مرجع{chuang} و تا امروز، هرچند رایانه‌های کوانتومی به شکلِِ تجاری وجود ندارند، اما با ایده‌های متعددی مبتنی بر فوتونیک، مدارهای ابررسانا یا غیره، در ابعادِ ۱۰ تا ۱۰۰ کیوبیت \زیرنویس{معادلِ کوانتومیِ بیت} ساخته‌شده‌اند. 
در سال ۲۰۱۹ نیز برای اولین‌بار، آزمایشِ برتریِ کوانتومی انجام شد. \مرجع{arute} این آزمایش به این معنی‌ست که محاسباتی انجام بگیرد که توسطِ هیچ رایانهٔ کلاسیکی ممکن نباشد.

با این‌همه، تا کنون تلاش‌های اندکی در بررسیِ مسائلِ هندسهٔ محاسباتی در سیاقِ رایانشِ کوانتومی صورت گرفته‌است. هرچند که به نظر می‌رسد که با توجه به خواصِ هندسیِ حالت‌های کوانتومی و قیودِ هندسی که مسائلِ هندسهٔ محاسباتی را از گونه‌های دیگرِ مسائلِ محاسباتی متمایز می‌کنند، رایانشِ کوانتومی کاربردهایی بسیار گسترده‌تر از آن‌چه تا کنون شناخته‌شده است در این حوزه داشته‌باشد.

\قسمت{تعریفِ موضوع}

موضوعِ این پایان‌نامه، در مرحلهٔ اول، بررسیِ کاربردهای شناخته‌شدهٔ الگوریتم‌های کوانتومی در هندسهٔ محاسباتی و در مرحلهٔ بعد معرفیِ الگوریتمِ جدیدی برای مسئلهٔ نقطه در چندضلعی است.

مسئلهٔ نقطه در چندضلعی، به این صورت است که چندضلعی‌ای در صفحه و یک نقطه در صفحه مشخص شده‌اند، مایلیم بدانیم که آیا نقطه درونِ چندضلعی قرار گرفته‌است یا بیرونِ آن. بیانِ ریاضیِ این مسئله در بخشِ \رجوع{مس:نقطه-در-چندضلعی} صورت خواهد گرفت.

\قسمت{اهمیتِ موضوع}

اهمیتِ این مسئله، از دو جهت قابلِ بررسی‌ست، یکی این‌که مسئلۀ نقطه در چندضلعی، در هندسهٔ محاسباتی مسئلهٔ مهمی‌ست و از سوی دیگر کاربردهای گوناگونی در سیستم‌های اطلاعاتِ جغرافیایی، گرافیکِ کامپیوتری و غیره دارد، از این رو تسریع در آن می‌تواند تفاوت‌های گوناگونی در فرایندهای مرسوم در این حوزه‌ها ایجاد کند.

از سوی دیگر، بررسیِ رایانه‌های کوانتومی و الگوریتم‌های آن‌ها هم‌اکنون از اهمیتِ زیادی برخوردار است، از این رو که قابلیت‌های این رایانه‌ها برای کاربردهای آیندهٔ نزدیک و دور شناسایی شود. رایانه‌های کنونی و تا آینده‌ای نزدیک، قابلیتِ حذفِ نوفه را ندارد و هر الگوریتمی که نسبت به آن تاب‌آوری داشته‌باشد، می‌تواند در ردیفِ اولین کاربردهای رایانشِ کوانتومی قرار بگیرد.

\قسمت{ساختار پایان‌نامه}

این پایان‌نامه در پنج فصل به این موضوع می‌پردازد به این ترتیب که پس از مقدمه، در فصلِ اول مفاهیمِ اولیهٔ رایانشِ کوانتومی و هندسهٔ محاسباتی مرور می‌شود و پس از آن در فصلِ مروری بر ادبیاتِ الگوریتم‌های کوانتومیِ هندسهٔ محاسباتی صورت می‌گیرد. سپس برای الگوریتمِ نقطه-در-چندضلعی الگوریتم‌های کلاسیک بررسی می‌شوند و در فصلِ چهارم الگوریتمِ کوانتومی‌ای تشریح می‌شود. درنهایت در فصلِ آخر به جمع‌بندی و مسیرهای پیشنهادی برای پژوهش‌های آتی پرداخته می‌شود.
