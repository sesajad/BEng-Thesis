

\فصل{مقدمه}

\قسمت{تعریف مسئله}


\قسمت{اهمیت موضوع}


\قسمت{ادبیات موضوع}


\قسمت{اهداف تحقیق}


\قسمت{ساختار پایان‌نامه}

این پایان‌نامه شامل پنج فصل است. 
فصل دوم دربرگیرنده‌ی تعاریف اولیه‌ی مرتبط با پایان‌نامه است. 
در فصل سوم مسئله‌ی دورهای ناهمگن و کارهای مرتبطی که در این زمینه انجام شده به تفصیل بیان می‌گردد. 
در فصل چهارم نتایج جدیدی که در این پایان‌نامه به دست آمده ارائه می‌گردد. در این فصل، مسئله‌ی درخت‌های ناهمگن در چهار شکل مختلف مورد بررسی قرار می‌گیرد. سپس نگاهی کوتاه به مسئله‌ی مسیرهای ناهمگن خواهیم داشت. در انتها با تغییر تابع هدف، به حل مسئله‌ی کمینه کردن حداکثر اندازه‌ی درخت‌ها می‌پردازیم.
فصل پنجم به نتیجه‌گیری و پیش‌نهادهایی برای کارهای آتی خواهد پرداخت.
