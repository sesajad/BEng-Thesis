

\فصل{مقدمه}

از ایدهٔ کامپیوترهای کوانتومی بیش از نیم قرن می‌گذرد اما عمدهٔ توجه به آن پس از دو الگوریتمِ مشهورِ شور و گروور در دههٔ نودِ میلادی بوده‌است. از آن روز تلاشِ بی‌وقفه‌ای برای یافتنِ الگوریتم‌های جدید و کاربردهای جدید از الگوریتم‌های قدیمیِ رایانشِ کوانتومی ادامه دارد. اما بخشِ اندکی از این تلاش‌ها در حوزهٔ هندسهٔ محاسباتی بوده‌است، با این وجود به نظر می‌رسد که با توجه به خواصِ هندسیِ حالت‌های کوانتومی و قیودِ هندسی که مسائلِ هندسهٔ محاسباتی را از گونه‌های دیگرِ مسائلِ محاسباتی متمایز می‌کنند، رایانشِ کوانتومی کاربردهایی بسیار گسترده‌تر از آن‌چه تا کنون شناخته‌شده است در این حوزه داشته‌باشد.

\قسمت{تعریف مسئله}
مسئلهٔ این تز، در مرحلهٔ اول، بررسیِ کاربردهای الگوریتم‌های کوانتومی در هندسهٔ محاسباتی و در مرحلهٔ بعد  معرفیِ الگوریتمِ جدیدی برای مسئلهٔ نقطه-در-چندضلعی است.
هرچند که برای مسئلهٔ نقطه‌-در-چندضلعی الگوریتمِ خطی‌ای وجود دارد و امکانِ تسریعِ آن به شکلِ کلاسیک وجود ندارد اما به علتِ استفاده‌ٔ گستردهٔ آن در گرافیکِ کامپیوتری، بهبودِ سرعتِ آن به شکلِ تکنیکال نیز همواره مورد توجهِ افراد بوده‌است.
همچنین در ادبیاتِ الگوریتم‌های کوانتومی در هندسهٔ محاسباتی، هرچند مسئله‌های بسیاری مورد بررسی قرار گرفته اما این مسئله موردِ بررسی قرار نگرفته و از سوی دیگر، ایدهٔ اکثریتِ تسریع‌های کوانتومی در هندسهٔ محاسباتی مبتنی بر جست‌وجوی کوانتومی بوده‌است که منتجِ به تسریعِ حداکثر مربعی می‌شود.

\قسمت{ساختار پایان‌نامه}

این پایان‌نامه در پنج فصل به این موضوع می‌پردازد به این ترتیب که پس از مقدمه، در فصلِ اول مفاهیمِ اولیهٔ رایانشِ کوانتومی و هندسهٔ محاسباتی مرور می‌شود و پس از آن در فصلِ مروری بر ادبیاتِ الگوریتم‌های کوانتومیِ هندسهٔ محاسباتی صورت می‌گیرد. سپس برای الگوریتمِ نقطه-در-چندضلعی الگوریتم‌های کلاسیک بررسی می‌شوند و در فصلِ چهارم الگوریتمِ کوانتومی‌ای تشریح می‌شود. درنهایت در فصلِ آخر به جمع‌بندی و مسیرهای پیشنهادی برای پژوهش‌های آتی پرداخته می‌شود.
