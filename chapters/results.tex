
\فصل{بحث و نتایجِ نو}

\قسمت{معرفیِ یک الگوریتمِ کوانتومی}

می‌بینیم صورت‌بندیِ ایدهٔ دوم که در فصلِ پیشین مطرح شد، شباهتِ زیادی به مسئلهٔ مربوط به الگوریتمِ دویچ-جوزا دارد، اما وجودِ وزن‌ها، امکانِ استفاده از الگوریتمِ دویچ‌جوزا را به ما نمی‌دهد.

اما می‌دانیم اگر حالتی به شکلِ زیر داشته‌باشیم
\begin{equation}
    \ket{\phi_1} := \frac{1}{\sqrt{N-1}} \sum_{q_iq_{i+1} \in \mathrm{segments}} \kappa \theta_i s_i \ket{q_i q_{i+1}} 
\end{equation}

که در آن $\kappa$ یک ضریب برای بهنجارسازی‌ست، آنگاه با استفاده از تبدیلِ فوریه، می‌توانیم به حالتی برسیم که دامنهٔ حالتِ 
$\ket{0}$
یا همان عنصرِ همانیِ گروه، در آن به شکلِ زیر باشد
\begin{equation}
    \bra{0} \mathrm{QFT} \ket{\psi} =  \frac{1}{N-1} \sum_{q_iq_{i+1} \in \mathrm{segments}} \kappa \theta_i s_i 
\end{equation}
که از آن‌جا که برای هرتبدیلِ فوریه‌ای این مهم برقرار است، می‌توان به جای 
$\mathrm{QFT}$
قرار داد
$H^{\otimes \log(N - 1)}$
و فرض بگیریم که $N-1$ توانی از دو است.

پس برای احتمالِ اندازه‌گیریِ 
$0$
پس از تبدیلِ فوریه داریم
\begin{equation}
    \label{eq:base-qalg-prob}
    \Pr(\text{اندازه‌گیریِ $0$}) = \abs{\bra{0} H^{\otimes \log(N - 1)} \ket{\psi}}^2 = \begin{cases}
    \frac{4\kappa^2\pi^2}{(N - 1)^2} & \text{نقطه داخلِ چندضلعی‌ست} \\
    0 & \text{نقطه بیرونِ چندضلعی‌ست}
    \end{cases}
\end{equation}
اما حالا ضریبِ $\kappa$ مربوط به فرایندِ تولیدِ این حالت است. اگر فرض کنیم ابتدا حالتی بسازیم که

\begin{equation}
    \ket{\phi_0} = \frac{1}{\sqrt{N-1}} \sum_{q_iq_{i+1} \in \mathrm{segments}} \theta_i s_i \ket{q_i q_{i+1}}
\end{equation}

سپس به سادگی با استفاده اضافه کردنِ کیوبیت، از جعبه‌سیاه‌ها و تعمیمِ مدارهای کلاسیک، می‌توانیم به حالتِ زیر برسیم

\begin{equation}
    \ket{\phi_0} = \frac{1}{\sqrt{N-1}} \sum_{q_iq_{i+1} \in \mathrm{segments}} \ket{q_i q_{i+1}}\ket{\arcsin{\kappa\theta_i}}\ket{s_i}
\end{equation}

و پس از آن با استفاده از ایده‌هایی مرسوم، نظیرِ ایده‌های استفاده شده در بخشِ \رجوع{بخش:گروور} به حالتِ $\ket{\phi_1}$ رسید.
\زیرنویس{ایدهٔ تبدیلِ $s_i$ به 
$(-1)^{s_i}$
دقیقاً مشابهِ آن‌چیزی‌ست که در بخشِ مذکور بحث شد، همچنین برای قسمتِ 
$\theta_i$
یک فرضِ طبیعی این است که مقدارِ 
$\arcsin(\kappa\theta_i)$
در چندکیوبیت به شکلِ دودویی ذخیره شده‌است و آن را می‌توان به شکلِ
$\overline{r_{2^d} r_{2^{d-1}}\dots r_2 r_1}$
که یکان و دوگان و الی آخر باشند. سپس، یک گیتِ شناخته‌شده و قابلِ پیاده‌سازی که به طورِ معمول در مدارهای کوانتومی نظیرِ تبدیلِ فوریه برای گروه‌های عددی استفاده می‌شود، گیتِ 
$\mathrm{C-}R_{x}$
است که عملکردِ آن به شکلِ زیر است
\begin{equation}
    \begin{cases} \mathrm{C-}R_{x}\ket{1}\ket{0} = \ket{1}(\cos(x)\ket{0} + \sin(x)\ket{1}) \\
    \mathrm{C-}R_{x}\ket{1}\ket{0} = \ket{0}\ket{0} \end{cases}
\end{equation}

حالا با در دست داشتنِ مدارهایی از این جنس، می‌توان حالتِ مذکور را تدارک دید.
یک نکتهٔ مهم فرایندِ پاک کردنِ اطلاعات کیوبیت‌های مورداستفاده 
$\theta_i$ و $s_i$ 
که با استفادهٔ دوباره از جعبه‌سیاه ممکن می‌شود و کیوبیت‌های مذکور
به حالتِ صفر و جدا می‌روند و می‌توان آن‌ها را از فرایند حذف کرد.}

پس با توجه به نکاتِ گفته‌شده لازم است که
$\kappa \theta_i \le 1$
که نتیجه می‌دهد بدیهی‌ترین انتخاب 
$\kappa = \frac{1}{\pi}$
باشد زیرا که زاویهٔ یک پاره‌خط در مقابلِ یک نقطه حداکثر به نیم‌صفحه می‌رسد.
در این صورت این احتمال نیز به شکلِ 
$\O{\frac{1}{N^2}}$
کوچک خواهدبود.

اما اگر قولی وجود داشته‌باشد که 
$\theta_i \le \frac{\gamma}{N} \in \O{\frac{1}{N}}$
که هم‌ارزِ این قول است که نقاطی که بررسی می‌کنیم به اضلاع بیش‌ازحد نزدیک نباشند و این فاصله از مرتبهٔ طولِ اضلاع باشد، آن‌گاه می‌توان $kappa$ را برابرِ 
$\frac{N}{\gamma \pi}$
قرار داد که در نتیجه احتمالِ تشخیصِ نقطه درونِ چندضلعی در معادلهٔ \رجوع{eq:base-qalg-prob} برابر با عددی ثابت خواهدشد که این یعنی با در این حالت تنها با یک پرسش می‌توان با خطای محدود به پاسخِ مسئله رسید.

این الگوریتم را می‌توان به شکلِ زیر بازنویسی کرد
\شروع{الگور}[آ]
\begin{latin}
\begin{lstlisting}
function IsPointInPolygonPromised(gamma: Double,
             coordsOfSeg: Hilbert(log(N-1) qubit × D qubit × D qubit) gate)
    index : log(N - 1) qubit state
    coordsStart : D qubit state
    coordsEnd : D qubit state
    arcsinTheta : D qubit state
    side : 1 qubit state

    function ClassicalCircuitForTheta(coordStart, coordEnd) = 
        arcsin(norm(coordStart - coordEnd) / norm((coordStart + coordEnd) / 2 - P) 
            * N / gamma / pi)
    function ClassicalCircuitForS(coordStart, coordEnd) = 
        sign(cross(P - (coordStart + coordEnd) / 2, coordEnd - coordStart).z)
        
    gate EncodedAngleFromP = quantum(ClassicalCircuitForTheta)
    gate SideFromP = quantum(ClassicalCircuitForS)
    
    // stage 1, initialization
    Initiate coordStart to 0
    Initiate coordEnd to 0
    Initiate arcsinTheta to 0
    Initiate side to 0
    for i : integer from 1 to log(N-1) {
        Initiate index[i] to 0
        Apply H on x[i]
    }
    
    // stage 2, pplying oracles
    Apply EncodedAngleFromP on coordStart, coordEnd, arcsinTheta
    Apply EncodedAngleFromS on coordStart, coordEnd, side
    // stage 3, Transforming oracle informations
    for i : integer from 1 to D {
        Apply C-R(2^(-i)) on arcsinTheta[i]
    }
    Apply Z on side
    // stage 4, pplying oracles again to remove data
    Apply EncodedAngleFromP on coordStart, coordEnd, arcsinTheta
    Apply EncodedAngleFromS on coordStart, coordEnd, side
    
    // stage 5, Hadamard transform and measurement
    for i : integer from 1 to log(N-1)
        Apply H on x[i]
        
    is_in : boolean = true
    for i : integer from 1 to log(N-1)
        result : boolean = Measure on x[i]
        if (result)
            is_in = false

    return is_in
\end{lstlisting}
\end{latin}
\پایان{الگور}
\قسمت{گسترشِ الگوریتم برای حالت‌های دیگر}

می‌دانیم که برای عملکردِ درستِ الگوریتم لازم است که احتمالی که در معادلهٔ \رجوع{eq:base-qalg-prob} افزایش یابد و به مقدارِ ثابتی برسد. از این رو، می‌توان از الگوریتمِ تقویتِ دامنه که در بخشِ \رجوع{بخش:گروور} تعریف شده‌است کمک بگیریم.
اگر کلِ فرایندِ الگوریتمِ قبل را تا پیش از اندازه‌گیری $G$ بنامیم، همچنین 
$\mathbb{P_T}$
را تصویر برروی عددِ $0$ باشد (که احتمالِ آن موردِ نظر است)، آن را تقویت کرد.
تعدادِ مراحلِ لازم برای این تقویت از مرتبهٔ
$\O{\frac{N}{\kappa}}$
خواهدبود که این نشان می‌دهد اگر $\kappa$ عدد ثابتی باشد، این الگوریتم هیچ تسریعی نمی‌تواند داشته‌باشد.

\قسمت{حدِ پایینِ دشمن‌گونه}

این‌طور که پیداست، مسئلهٔ نقطه در چندضلعی در حالتِ کلی نمی‌تواند تسریعی با استفاده از رایانشِ کوانتومی را تجربه کند. این موضوع به شکلِ تئوری نیز قابلِ بررسی‌ست.

تا به این‌جای بحث، محدودیتی برروی سادگی یا غیرِسادگیِ چندضلعی‌ها مشخص نشده و قابلِ حدس است که تمامِ بحث‌های گفته‌شده برروی هردو دستهٔ چندضلعی‌ها برقرار باشند. اما در این بخش، استدلالی برای حدِ پایینِ پرسش‌های کلاسیک و کوانتومیِ لازم برای حلِ مسئلهٔ مذکور بیان می‌شود که تنها برای چندضلعی‌های غیرساده معتبر است و درصورتِ محدودیتِ مسئله به چندضلعی‌های ساده، این حدود غیرمعتبر خواهندبود.

% TODO make this a theorem
برای بیانِ این حد از حدِ پایینِ دشمن‌گونه استفاده می‌کنیم که به این ترتیب است که اگر برای مسئله‌ای به شکلِ
$f: S \to \mathbb{Z}_2$
دسترسیِ الگوریتم به ورودی از طریقِ جعبه‌سیاه باشد؛ یعنی برای هر ورودیِ مسئله مانندِ $s \in S$ الگوریتم با استفاده از جعبه‌سیاهی مانندِ
$O_s$
به جوابِ مسئله برسد، و همچنین دو زیر مجموعهٔ دلخواهِ زیر را داشته‌باشیم

\begin{eqnarray}
    X \subseteq \{ s | f(s) = 1 \} \\
    Y \subseteq \{ s | f(s) = 0 \}
\end{eqnarray}
و رابطه‌ای به شکلِ
$R \subseteq X \times Y $
که 
\begin{equation}
    x R y \Leftrightarrow O_x(i) \ne O_y(i) \text{تنها برای یک مقدارِ $i$}
\end{equation}

از طرفِ دیگر می‌توانیم رابطهٔ $R_i$ را نیز به شکلی تعریف کنیم که
\begin{equation}
    x R_i y \Leftrightarrow O_x(i) \ne O_y(i) \text{ و } \forall j \ne i ~ O_x(j) = O_y(j)
\end{equation}
که در این صورت
\begin{equation}
    R = \bigcup_{i} R_i
\end{equation}
حالا اگر گرافِ دوبخشیِ معادل با $R$ را در نظر بگیریم، کمینه درجهٔ رئوسِ بخشِ $X$ و بخشِ $Y$ را به ترتیب 
$m$
و 
$m'$
بنامیم، از سوی دیگر، برای $R_i$ ها بیشینهٔ  درجهٔ رئوس را به شکل $l_i$ و $l'_i$ را تعریف کنیم و بگیریم
\begin{eqnarray}
    l := \max_i l_i \\
    l':= \max_i l'_i
\end{eqnarray}
آن‌گاه پیچیدگیِ محاسباتیِ پرسش های این مسئله از مرتبهٔ
$\Omega(\sqrt{\frac{m m'}{l l'}}$
خواهد بود. که بدونِ اثبات آن را خواهیم‌پذیرفت \مرجع{ambainis_2002} 


حالا برای استفاده از حدِ دشمن‌گونه، مسئلهٔ زیر را تعریف می‌کنیم

اگر یک $N-$ضلعیِ منتظم $Q$ را درنظر بگیریم که نقطهٔ $P$ مرکزِ آن باشد، حالا چندضلعیِ $Q'$ را با مقیاس کردنِ $Q$ به مرکزِ $P$ و با ضریبِ 
$\frac{1}{2}$
 و سپس قرینهٔ نقطه‌ای کردن آن حولِ $P$ بسازیم، آن‌گاه به ازای هر رشتهٔ $N$-بیتیِ $s$ یک چندضلعیِ 
 $Q^{(s)}$
 خواهیم داشت که رئوسِ آن به این ترتیب به دست می‌آیند
\begin{equation}
 q^{(s)}_i = \begin{cases}
  q_i & s_i = 1 \\
  q'_i & s_i = 0
 \end{cases}
\end{equation} 

حالا مسئلهٔ وجودِ نقطهٔ $P$ در چندضلعیِ $Q^{(s)}$ برابرِ مسئلهٔ زوج بودنِ وزنِ همینگِ $s$ خواهد بود. برای اثباتِ این برابری، می‌توان از استقرای ریاضی استفاده کرد به این ترتیب که به ازای $s=0$ این برابری به سادگی برقرار است و با تغییرِ هر بیت از $s$ می‌توان به سادگی نشان داد که همچنان برابری حفظ می‌شود و درنتیجه برای تمامِ رشته‌ها برقرار است.

از سوی دیگر، برای مسئلهٔ زوج بودنِ وزنِ همینگِ $s$، می‌توانیم از حدِ دشمن‌گونه به این ترتیب استفاده کنیم که $X$ همهٔ رشته‌ها با وزنِ زوج و $Y$ همهٔ رشته‌ها با وزنِ فرد باشند، آن‌گاه $m$ و $m'$ هردو برابر با طولِ رشته و برابر با $N$ خواهندبود و مقادیرِ $l$ و $l'$ نیز که مربوط همسایه‌هایی هستند که تنها در پرسشِ خاصِ $i$ (بخوانید بیتِ $i$ام) با هم تفاوت دارند برابر با $1$ خواهند بود، درنتیجه، این مسئله نیاز به 
$\Omega(N)$
پرسش خواهدداشت.

از آن‌جا که مسئلهٔ زوج بودنِ وزنِ همینگ قابلِ کاهش به وجودِ نقطه در چندضلعی‌ست پس حداقل پرسش برای مسئلهٔ نقطه در چندضلعی نیز برابرِ 
$\Omega(N)$
خواهدبود.

