
\فصل{نتیجه‌گیری}

در این پایان‌نامه سعی شد که به بررسیِ رایانشِ کوانتومی و هندسهٔ محاسباتی و تلاقیِ این دو حیطه پرداخته‌شود. آن‌چه به نظر می‌رسد این است که تلاش‌های کمی در ترکیبِ این دو حوزه صورت گرفته‌است. حال آن‌که به خاطرِ ارتباطِ گستردهٔ هندسه (به‌خصوص در ابعادِ بالا) و رایانشِ کوانتومی پتانسیلِ خوبی برای تسریع‌های کوانتومی در مسائلِ هندسی وجود دارد.
همچنین نظیرِ آن‌چه در این پایان‌نامه گفته شد، بسیاری از تسریع‌های کوانتومی در حضورِ قیدها و قول‌ها به دست می‌آیند که در این حوزه به خاطرِ ذاتِ هندسیِ مسائل، همواره قیدهایی برروی ورودی وجود خواهند داشت و این هم‌خوانی حتماً قابلِ استفاده خواهدبود.

آن‌چه ماحصلِ این پژوهش بوده‌است، به طورِ خاص برای مسئلهٔ نقطه در چندضلعی، به این ترتیب است که نشان داده‌شده هیچ الگوریتمِ کوانتومی‌ای نخواهدتوانست درحالتِ کلی، سریع‌تر از الگوریتم‌های کلاسیک به پاسخِ این مسئله دست پیدا کند. اما با اندکی تغییرِ مسئله و ایجادِ یک قول، مبتنی بر فاصله داشتنِ نقطه از اضلاع، یا حتی با قدری کاهشِ حساسیت نسبت به خطا در نزدیکیِ خطوطِ چندضلعی، می‌توان از الگوریتمِ پیشنهادشده استفاده کرد که از آن‌جا که مبتنی بر تبدیلِ فوریهٔ کوانتومی بوده‌است می‌تواند باعثِ تسریعِ فرا-چندجمله‌ای بشود و تعدادِ پرسش‌ها را تا $\Theta(1)$ و پیچیدگیِ زمانی را تا 
$\Theta(\log(n))$
کاهش دهد.

نکتهٔ حائزِ اهمیتِ دیگر این است که از این الگوریتم می‌تواند به مقدارِ دلخواهی خطا را کم و به پیچیدگیِ محاسباتی اضافه کند تا به دقت و سرعتِ الگوریتمِ کلاسیک برسد.

از حیثِ شیوهٔ عملکردِ الگوریتم نیز، با توجه به مرورِ ادبیات، تنها الگوریتمی خواهدبود که بر مبنای جست‌وجو و ولگشت عمل نمی‌کند و از تسریعِ تبدیلِ فوریهٔ کوانتومی استفاده می‌کند.

\قسمت{کارهای آتی}

در این پژوهش جای خالیِ شبیه‌سازی و نمایشِ خروجی‌ها برای مشاهدهٔ شرایطِ قول و مقدارِ خطا وجود دارد. همچنین بررسیِ الگوریتم‌های تقریبیِ کلاسیک و طراحیِ الگوریتم‌های کلاسیک برای همان شرایطِ قول می‌تواند منجر به مقایسهٔ دقیق‌تری بینِ راه‌حلِ کلاسیک و کوانتومی در این مسئله بشود. از سوی دیگر، بررسیِ کاربردهای مسئلهٔ قولی در هندسهٔ محاسباتی و حوزه‌های دیگری نظیرِ گرافیکِ کامپیوتری همچنان موردِ سؤال است.

فراتر از این، همچنان بسیاری از مسائل در هندسهٔ محاسباتی هستند که هیچ راه‌حلِ کوانتومی‌ای برای آن‌ها پیشنهاد نشده و از سوی دیگر، مسائلی که راه‌حل یا حدِ کوانتومی دارند نیز، نیازمندِ جمع‌بندی و تدوین هستند تا ابزاری یکپارچه شوند. برای مثال، بررسیِ شیوهٔ ورودی گرفتنِ اشکالِ کوانتومی یا پیدا کردنِ فرایندهای مشترک در الگوریتم‌های کوانتومیِ این حوزه، از موضوعاتِ ارزشمند برای پژوهش‌های آتی هستند.